\documentclass{article}
\usepackage[utf8]{inputenc}
\usepackage{geometry} %
\usepackage{ragged2e}%
\usepackage{tabularx, makecell, booktabs, caption}
\usepackage{amsmath}

\begin{document}
	\section{Testspezifikationen}
	\subsection{FRAM}
		\begin{tabular}{|p{3.2cm}|p{5.5cm}|p{4cm}| p{1.6cm} |}
			\hline
				\textbf{Funktionsname} & \textbf{Beschreibung} & \textbf{Ergebnis} & \textbf{Status} \\
			\hline
				\texttt{FRAM\char`_init} & Write Enable Latch (WEL) setzen und Status Register auslesen & Status Register ist 2 & Funktional \\
			\hline
				\texttt{FRAM\char`_test} & FRAM initialisieren und and Addresse \texttt{0x0000} "Test" schreiben und diese Adresse auslesen & Geschriebene Nachricht ist identisch mit gelesener Nachricht & Funktional\\
			\hline
				\texttt{Test\char`_FRAM} & Essentieller Teil des Localization structs mit Testwerten initialisieren, serialisieren und in FRAM schreiben. Anschließend wieder auslesen und in das struct Objekt zurück deserialisieren & - & - \\
			\hline
		\end{tabular}
	\subsection{LEDs}
		\begin{tabular}{|p{3.2cm}|p{5.5cm}|p{4cm}| p{1.6cm} |}
			\hline
				\textbf{Funktionsname} & \textbf{Beschreibung} & \textbf{Ergebnis} & \textbf{Status} \\
			\hline
				 \texttt{Test\char`_LED} & Alle Status LEDs werden über den Switch-Button erst an und danach ausgeschaltet & - & - \\
			\hline
		\end{tabular}
		\subsection{Buttons/Switch}
	\begin{tabular}{|p{3.2cm}|p{5.5cm}|p{4cm}| p{1.6cm} |}
		\hline
			\textbf{Funktionsname} & \textbf{Beschreibung} & \textbf{Ergebnis} & \textbf{Status} \\
		\hline
			 \texttt{Test\char`_Button} & Für jeden Button wird nacheinander eine steigende und fallende Flanke abgefragt, welche durch eine Kontroll-LED repräsentiert wird & - & - \\
		\hline
	\end{tabular}
	\subsection{Motor}
		\begin{tabular}{|p{3.2cm}|p{5.5cm}|p{4cm}| p{1.6cm} |}
			\hline
				\textbf{Funktionsname} & \textbf{Beschreibung} & \textbf{Ergebnis} & \textbf{Status} \\
			\hline
				\texttt{Test\char`_Motor} & Zunächst Fehlerstatus auslesen. Über den Switch-Button werden mehrere Schritte eingeleitet: - Links-/Rechtslauf mit Speed1 und anschließendem Auslesen der Richtung - Stufenweises Erhöhen der Drehzahl bis auf Speed2 und Auslesen der jeweils gemessenen Drehzahl & - & - \\
			\hline
		\end{tabular}
	\subsection{IFS204 (Endschalter)}
		\begin{tabular}{|p{3.2cm}|p{5.5cm}|p{4cm}| p{1.6cm} |}
			\hline
				\textbf{Funktionsname} & \textbf{Beschreibung} & \textbf{Ergebnis} & \textbf{Status} \\
			\hline
				 \texttt{Endswitch\char`_detected} & Auslesen der Endschalter Zustände & Ausgelesene Zustände entsprechen den Endschalter Zuständen & Funktional \\
			\hline
				\texttt{Test\char`_endswitch} & Der Motor fährt die Endschalter an, und ändert die Richtung, sobald diese aktiviert werden & - & - \\
			 \hline
		\end{tabular}
	\subsection{OGD580 (Abstandssensor)}
		\begin{tabular}{|p{3.2cm}|p{5.5cm}|p{4cm}| p{1.6cm} |}
			\hline
				\textbf{Funktionsname} & \textbf{Beschreibung} & \textbf{Ergebnis} & \textbf{Status} \\
			\hline
				\texttt{-} & Analoger Abstandswert wird ausgelesen und mit dem Angezeigten Wert des Displays verglichen & - & - \\
			\hline
				\texttt{-} & Linearführung wird um eine gewisse Distanz bewegt. Distanz sollte der Differenz aus End- und Startposition entsprechen & - & - \\
			\hline
		\end{tabular}
	\subsection{WSWD (Windsensor)}
		\begin{tabular}{|p{3.2cm}|p{5.5cm}|p{4cm}| p{1.6cm} |}
			\hline
				\textbf{Funktionsname} & \textbf{Beschreibung} & \textbf{Ergebnis} & \textbf{Status} \\
			\hline
				\texttt{-} & Abfragen der Seriennummer über RS485 und Überprüfung dieser & - & - \\
			\hline
		\end{tabular}
	\subsection{Stromsensor}
		\begin{tabular}{|p{3.2cm}|p{5.5cm}|p{4cm}| p{1.6cm} |}
			\hline
				\textbf{Funktionsname} & \textbf{Beschreibung} & \textbf{Ergebnis} & \textbf{Status} \\
			\hline
				\texttt{-} & Auslesen des Stroms während der Motor nicht in Bewegung ist. Ausgelesene Spannung sollte der halben Versorgungsspannung des Sensors entsprechen (ca. 1.6V) & - & - \\
			\hline
				\texttt{-} & Motor wird über die Stromvorgabe angesteuert. Gemessener Strom sollte dem vorgegebenen Strom entsprechen. & - & - \\
			\hline
		\end{tabular} 
\end{document} 
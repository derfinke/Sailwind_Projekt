\nonumsection{Kurzfassung}
Die Bachelorarbeit beschäftigt sich mit der Konzeption, Implementierung und Testung einer \acfi{OCPP} Schnittstelle die in ein bestehendes Softwaresystem, dem sog. Tankautomaten eingefügt werden soll. Über den Tankautomat können Zapfsäulen auf Unternehmenstankstellen gesteuert und verwaltet werden. Damit diese Funktionalität zukünftig auch für elektrische Fahrzeuge verfügbar ist, soll mithilfe des Protokolls eine Kompatibilität für Ladesäulen geschaffen werden. Die Funktionsfähigkeit der Umsetzung soll daraufhin mithilfe von Tests überprüft werden.\\
\noindent Die Konzeption erfolgt durch die Analyse des benötigten Protokollumfangs, die Anpassung bestehender Systeme und der Erstellung eines Plans zum Vorgehen, um damit die Umsetzung zu gestalten. Wie die Umsetzung erfolgte wird im Implementierungsteil der Arbeit erläutert. Abschließend erfolgte eine Testplanung und Durchführung, um den Erfolg der Umsetzung zu bestimmen. Des Weiteren wurden andere Testverfahren vorgeschlagen, um die Schnittstelle noch robuster zu gestalten. \\
\noindent In dieser Arbeit wird also ein Überblick über das \acs{OCPP} Protokoll und wie dieses mit einem möglichst kleinen Umfang genutzt werden kann gegeben. Als Ergebnis wurde festgehalten, dass das Protokoll erfolgreich in die Firmware integriert wurde und die Tests an zwei von drei Testladesäulen erfolgreich waren.
\thispagestyle{empty}
\newpage

\section*{Abstract}
The bachelor thesis deals with the conception, implementation and testing of an \ac{OCPP} interface that is to be integrated into an existing software system, the so-called refuelling terminal. The refuelling terminal can be used to control and manage fuel pumps on company property. To ensure that this functionality is also available for electric vehicles, the protocol is to be used to create a compatibility for charging stations. The functionality of this implementation will then be confirmed with the help of tests.\\\
\noindent During the conecption chapter the needed functionality of the protocol is analyzed by determining the required scope of the protocol, adapting existing systems and creating a plan of action in order to design the implementation. How this plan of action was then carried out is described during the implmentation chapter. After that, test planning and execution is carried out to determine the success of the implementation. Other test procedures were also proposed to make the interface implementation even more robust. \\
\noindent This paper therefore provides an overview of the \acs{OCPP} protocol and how it can be used to manage charging stations with the smallest possible scope. As a result, the protocol was successfully integrated into the firmware and the tests where successfully carried out on two of the three testing charging stations.

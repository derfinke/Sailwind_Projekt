\section{Hardware}
Hier erwähnen was Platine alles für Aufgaben hat
\subsection{Planung}
\subsection{Analyze der bestehenden Hardware Elemente}
Um eine geeignete Steuerplatine zu entwerfen, bedarf es einer Analyze der anzusteurenden Hardware Komponenten. Diese wurden bereits vorrausgehend festgelegt und wurden übernommen. Diese können in die Kategorien Sensoren und Aktoren gruppiert werden. Dabei wurden die benötigte Spannung, der Stromverbrauch und die Schnittstellen betrachtet. Durch die Spannung und die Stromaufnahme kann später die benötigte Spurenbreite auf der Platine bestimmt werden. Die Schnittstellen geben an wie mit dem jeweiligen Gerät kommuniziert werden kann und welche davon benötigt werden.\\

\subsubsection{Sensoren}
Zu den Sensoren zählen die folgenden Komponenten:
\begin{itemize}
	\item Induktiver Endschalter: IFM IFS204
	\item Optischer Abstandssensor: IFM OGD580
	\item Anemometer: MESA WSWD
	\item Druckkraft-Sensor: Burster 8532
\end{itemize}

\noindent\textbf{Induktiver Endschalter}\newline
Zwei der IFM IFS204 Endschalter sind im Design mit eingebaut. Sie funktioneren als PNP-Schließkontakt und geben an einem Ausgang ein 24V high Signal aus, sobald sie ausgelöst werden. Dabei hat jeder Endschalter eine typische Stromaufnahme von <10mA und eine Betriebsspannung von 9-30V.\\

\noindent\textbf{Optischer Abstandssensor}\newline
Der Abstandssensor OGD580 funktioniert über einen Laser der vom Gerät ausgehend über eine reflektive Fläche zu diesem zrückgeworfen wird. Der Abstandsensor verfügt über ein Display das den gemessen Abstand anzeigt und über den das Gerät konfigurierbar ist. Er hat eine typische Stromaufnahme von ???mA und ebenfalls eine Betriebspannung von 9-30V. Der Abstand wird über einen Digitalen Ausgang, dem sog. IO-Link ausgegeben. Da dieser typischerweise nur in der Automobilbranche zum Einsatz kommt, wurde ein zusätzlicher IO-Link Konverter hinzugefügt. Der EIO104 konvertiert die digitale IO-Link Schnittstelle zu einer Analogen 4-20mA Schnittstelle mit der einfacher Umgegangen werden kann. Dabei kommt ein zusätzlicher Stromverbrauch von ???mA hinzu.\\

\noindent\textbf{Anemometer}\\
Das WSWD Anemometer wird genutzt um die Windrichtung als auch die Windgeschwindigkeit zu messen. Dieser basiert ebenfalls auf einer 24V Versorgungsspannung und hat je nach Konfiguration und Ausführung eine Stromaufnahme von 120mA. Er besitzt ebenfalls ein Heizelement um ihn bei sehr niedrigen Temperaturen nutzen zu können. Dieses wird allerdings im Einsatzszenario und im Prototyp nicht benötigt. Die Schnittstellen des Anenometers sind abhängig von dessen ausführung. In der zum Einsatzkommenden Ausführung, dem WSWD1, kann neben einer Analogen Schnittstelle die Daten auch über eine Digitale RS485/RS422 Schnittstelle abgefragt werden. Die Werte können Analog entweder als ein 0/4-20/24mA Signal oder als ein 0/2-8/10V Signal ausgegeben werden. Die Digitale Schnittstelle kann neben der Datenausgabe auch zur Konfiguration des Gerätes genutzt werden und bietet eine Vielzahl an Protokollen zur Kommunikation an.\\

\noindent\textbf{Druckkraft-Sensor}\newline
Der Burster Druckkraft-Sensor war der einzige Sensor der zum Zeitpunkt der Arbeit noch nicht Bestellt wurde. Dieser wurde aber dennoch mit eingeplant. Der Druckkraft Sensor kann ebenfalls mit den typischen 24V betrieben werden und hat dabei eine Stromaufnahme von ca. 12,5mA. Die Messwerte werden hier über eine Analoge 0-10V Schnittstelle übertragen.\\
\subsubsection{Aktoren}
Zu den Aktoren zählen die folgenden Komponenten:
\begin{itemize}
	\item Gleichstrommotor: Dunkermotoren BG 45x30 SI
	\item Externe Relais
\end{itemize}

\noindent\textbf{Gleichstrommotor}\newline
Der Dunkermotor BG 45x30 SI Gleichstrommotor ist das Kernelement des Aufbaus. Da der Motor eine interne Regelung besitzt hat er eine getrennte Leistungs- und Logikversorgung. Der Motor an sich wird dabei über den Leistungsteil bestromt, während über den Logikteil dieser gesteuert werden kann und Feedback bereitstellt. Die Betriebsspannung der Leistungs- und Logikversorgung sind 24V, wobei der Motor einen maximal zulässigen Dauerstrom von 3,8A ausgesetzt sein darf. Die Logikversorung hat eine Stromaufnahme von 100mA. Die Kommunikation mit dem Motor findet über vier Digitale- und einen Analogen Eingang statt. Der Motor stellt Feedback zum aktuellen Status über drei Ausgänge bereit. Diese geben die Drehrichtung, aktuelle Störungen und die Drehgeschwindigkeit des Motors an.\\
% Please add the following required packages to your document preamble:
% \usepackage{graphicx}
\begin{table}[H]
	\centering
		\begin{tabular}{|c|c|c|}
			\hline
			\textbf{Eingang 1 (IN1)} & \textbf{Eingang 0 (IN0)} & \textbf{Funktion}      \\ \hline
			0                        & 0                        & Motor aus              \\ \hline
			0                        & 1                        & Linkslauf              \\ \hline
			1                        & 0                        & Rechtslauf             \\ \hline
			1                        & 1                        & Stopp mit Haltemoment  \\ \hline
			\textbf{Eingang 3 (IN3)} & \textbf{Eingang 2 (IN2)} &                        \\ \hline
			0                        & 0                        & Drehzahlvorgabe Analog \\ \hline
			0                        & 1                        & Stromvorgabe Analog    \\ \hline
			1                        & 0                        & Geschwindigkeit 1      \\ \hline
			1                        & 1                        & Geschwindigkeit 2      \\ \hline
		\end{tabular}%
	\caption{Eingänge und Funktionen des BG 45x30 SI}
	\label{tab:digitale_Eingaenge}
\end{table}
Hier oder später noch Ein und Ausgänge definieren in Tabelle bzw. in Text erwähnen
\\

\noindent\textbf{Externe Relais}\newline
Zum Zeitpunkt der Arbeit gab es noch keinen konkreten Verwendungszweck der externen relais diese wurden für zusätzliche Funktionalitäten dennoch mit eingeplant und sollten über ein 24V Signal geschalten werden.

\subsubsection{Zusammenfassung und Übersicht}
% Please add the following required packages to your document preamble:
% \usepackage{graphicx}
\begin{table}[H]
	\centering
	\resizebox{\textwidth}{!}{%
		\begin{tabular}{|l|lllllllllll|}
			\hline
			\textbf{Element} & \multicolumn{11}{l|}{\textbf{Pinout}} \\ \hline
			\textbf{Endschalter 1} & \multicolumn{1}{l|}{24V} & \multicolumn{1}{l|}{GND} & \multicolumn{1}{l|}{OUT} & \multicolumn{8}{l|}{} \\ \hline
			\textbf{Endschalter 2} & \multicolumn{1}{l|}{24V} & \multicolumn{1}{l|}{GND} & \multicolumn{1}{l|}{OUT} & \multicolumn{8}{l|}{} \\ \hline
			\textbf{Abstandssensor} & \multicolumn{1}{l|}{24V} & \multicolumn{1}{l|}{GND} & \multicolumn{1}{l|}{IO-Link} & \multicolumn{8}{l|}{} \\ \hline
			\textbf{IO-Link Konveter} & \multicolumn{1}{l|}{24V} & \multicolumn{1}{l|}{GND} & \multicolumn{1}{l|}{AOUT} & \multicolumn{8}{l|}{} \\ \hline
			\textbf{Anemometer} & \multicolumn{1}{l|}{24V} & \multicolumn{1}{l|}{GND} & \multicolumn{1}{l|}{AOUT} & \multicolumn{1}{l|}{AOUT} & \multicolumn{1}{l|}{AGND} & \multicolumn{1}{l|}{RS485} & \multicolumn{1}{l|}{RS485} & \multicolumn{4}{l|}{} \\ \hline
			\textbf{Druckkraftsensor} & \multicolumn{1}{l|}{24V} & \multicolumn{1}{l|}{GND} & \multicolumn{1}{l|}{AOUT} & \multicolumn{1}{l|}{AGND} & \multicolumn{7}{l|}{} \\ \hline
			\textbf{Motor Leistung} & \multicolumn{1}{l|}{24V} & \multicolumn{1}{l|}{GND} & \multicolumn{9}{l|}{} \\ \hline
			\textbf{Motor Logik} & \multicolumn{1}{l|}{24V} & \multicolumn{1}{l|}{GND} & \multicolumn{1}{l|}{IN0} & \multicolumn{1}{l|}{IN1} & \multicolumn{1}{l|}{IN2} & \multicolumn{1}{l|}{IN3} & \multicolumn{1}{l|}{AIN} & \multicolumn{1}{l|}{AGND} & \multicolumn{1}{l|}{OUT1} & \multicolumn{1}{l|}{OUT2} & OUT3 \\ \hline
			\textbf{Externes Relais 1} & \multicolumn{1}{l|}{OUT} & \multicolumn{1}{l|}{GND} & \multicolumn{9}{l|}{} \\ \hline
			\textbf{Externes Relais 2} & \multicolumn{1}{l|}{OUT} & \multicolumn{1}{l|}{GND} & \multicolumn{9}{l|}{} \\ \hline
		\end{tabular}%
	}
	\caption{Funktionen digitaler Ausgänge und analoger Eingang}
	\label{tab:my-table}
\end{table}
\subsection{Auswahl der neuen Bauteile}
Da nun klar ist welche Anforderungen durch die anzuschließenden Geräte bestehen, kann auf Basis dieser nun weiterverfahren werden. Dabei soll im folgenden ein geeigneter Mikrocontroller, sowie die nötigen Platinenkomponenten ausgewählt werden, um mit den Hardwarekomponenten zu kommunizieren und diese mit Strom zu versorgen..
\subsubsection{Platinen Entwurf}
\subsection{Probleme}
\section{Hardware}
\subsection{Planung}
\subsection{Analyze der bestehenden Hardware Elemente}
Um eine geeignete Steuerplatine zu entwerfen, bedarf es einer Analyze der anzusteurenden Hardware Komponenten. Diese wurden bereits vorrausgehend festgelegt und wurden übernommen. Diese können in die Kategorien Sensoren und Aktoren gruppiert werden. Dabei wurden die benötigte Spannung, der Stromverbrauch und die Schnittstellen betrachtet. Durch die Spannung und die Stromaufnahme kann später die benötigte Spurenbreite auf der Platine bestimmt werden. Die Schnittstellen geben an wie mit dem jeweiligen Gerät kommuniziert werden kann und welche davon benötigt werden.\\

\subsubsection{Sensoren}
Zu den Sensoren zählen die folgenden Komponenten:
\begin{itemize}
	\item Induktiver Endschalter: IFM IFS204
	\item Optischer Abstandssensor: IFM OGD580
	\item Anemometer: MESA WSWD
	\item Druckkraftsensor: Burster 8532
\end{itemize}

\subsubsection{Aktoren}
\subsection{Probleme}
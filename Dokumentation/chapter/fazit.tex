\section{Fazit und Ausblick}
Abschließend kann für diese Arbeit festgehalten werden, dass die neue Schnittstelle erfolgreich implementiert und getestet wurde. Der Tankautomat ist jetzt in der Lage über das OCPP Protokoll Verbindungen zu Ladesäulen herzustellen und Ladevorgänge zu verwalten. Dabei konnte während der Konzeption durch die Analyse der Abläufe die benötigten Nachrichten und Konfigurationsparameter ermittelt werden. Mit diesem Vorgehen wurde für eine kürzere Umsetzungszeit und weniger Aufwand gesorgt. Aufgrund des kleinen benötigten Umfanges und der unbefriedigenden Auswahl an fertigen Lösungen, wurde sich für eine Selbstentwicklung des Protokolls entschieden, die auf der Basis der Websocketpp Bibliothek aufgebaut wurde. Durch die komplette eigene Entwicklung der Erweiterung konnte ebenfalls dafür gesorgt werden, dass zukünftig auch eine Verschlüsselung der Kommunikation und die Erweiterung des Protokollumfang einfach zu gestalten sind. Die Ergänzung der Weboberflächen Felder wurde genutzt um einfach die Ladesäulen identifizieren zu können und hilft dem Kunden die Ladesäulen individueller zu konfigurieren. Mit dieser Basis konnte der Umsetzungsplan entworfen werden. Dieser konnte dann während der Implementierung angewandt und umgesetzt werden. Dabei gab es allerdings unvorhergesehene Probleme mit den Protokollabläufen der OCPP2.0.1 Version, die aber Dank der Tests auffielen und schnell behoben werden konnten. Abschließend wurde mithilfe der Tests, am Großteil der Ladesäulen, verifiziert, dass die Anforderungen erfüllt wurden. Es muss aber in Folge der Komplikationen mit der Wallbe Ladesäule noch weiter nach dem Grund der Inkompatibiliät gesucht werden. In Zuge dessen sollten die Tests in Zukunft weiter ausgebaut und automatisiert werden, damit auch weitere Sonder- und Randfälle abgedeckt sind. Dafür wurden Empfehlungen und mögliche Vorgehensweisen dargelegt die in der Zukunft vom Entwicklerteam umgesetzt werden. 
\newline


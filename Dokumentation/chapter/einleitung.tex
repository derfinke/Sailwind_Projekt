\section{Einleitung}
Durch die zunehmende Elektrifizierung des Verkehrssektors gewinnt die Ladeinfrastruktur für elektrische Fahrzeuge immer mehr an Bedeutung. Um dem Wandel in der Mobilitätslandschaft gerecht zu werden, werden daher immer mehr E-Ladesäulen in unterschiedlichsten Umgebungen installiert. Dazu gehören z. B. Tankstellen, Autobahnraststätten und Firmenparkplätze. Das \acfi{OCPP} hat sich dabei als weitverbreiteter und viel unterstützter Standard zur Kommunikation mit Verwaltungssystemen etabliert (vgl. \cite{OCPP-Chronicles_popularity}, S.5).
%Parkplätze allgemein passen nicht wirklich zur Einleitung bzw. Zum folgenden Satz.
\subsection{Ausgangssituation}
Die Konzept Informationssysteme GmbH (kurz: Konzept) entwickelt für einen Kunden im Zuge einer neuen Entwicklung, die Software zum Produkt Tankautomat (im weiteren Verlauf auch Terminal und \ac{BTS3} genannt). Über diesen können sich Mitarbeiter per Touchdisplay und/oder mittels eines \acfi{RFID}-Tags authentifizieren, um Zugriff auf Tankstationen zu erhalten. Dabei ist die Freigabe nicht nur auf die klassischen Kraftstoffstationen (z. B. Benzin, Diesel) beschränkt, sondern kann auch auf Waschanlagen oder Garagen erfolgen. Die Hauptaufgabe der Software des Tankautomaten ist es also, den Anwender nach einer Berechtigungsprüfung freizuschalten und zur Tankfreigabe zu leiten (vgl. \cite{Lastenheft_Software}, S.5). Dabei werden die Tankvorgänge z. B. aus Regulatorischen Gründen Gesteuert und Protokolliert \newline
\noindent Ein weiterer Teil des Projektes ist die neue Cloud Weboberfläche. Hier haben Administratoren die Möglichkeit Einstellungen zu den Fahrern, Fahrzeugen und Stationen zu treffen. Ebenfalls können hier auch Tankvorgänge eingesehen oder Software Updates durchgeführt werden. Die Entwicklung der Oberfläche liegt nicht bei Konzept, sondern wird von einem Projektpartner durchgeführt.\newline
\noindent Die neue Version des Tankautomaten soll hauptsächlich auf Unternehmenstankstellen z. B. für Busunternehmen oder Unternehmensparkplätzen eingesetzt werden.
\subsection{Ziel}
Um den Tankautomaten für die Zukunft vorzubereiten und die wachsende Nachfrage der E-Mobilität zu bewältigen, soll dieser um eine Schnittstelle zur Kommunikation mit E-Ladesäulen erweitert werden. Hierfür soll das bereits erwähnte OCPP genutzt werden. Die Hauptaufgabe in der vorliegenden Arbeit ist es, dieses Protokoll in der Version 1.6 und der neusten Version 2.0.1 in die Software des Tankautomaten einzubinden. Dadurch soll garantiert werden, ein breites Spektrum an Ladesäulen zu unterstützen. Dabei sollen die bereits existierenden Funktionen und Abläufe an Ladevorgänge adaptiert werden. Zusätzlich sollen im Zuge dessen Tests entworfen werden, um die Funktionalität der neuen Schnittstelle zu garantieren und auch zukünftig überprüfen zu können.

%Maybe noch mehr den Wirtschaftlichen Aspekt erwähnen.
\newpage



